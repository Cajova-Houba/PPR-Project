\documentclass[11pt,a4paper]{scrartcl}
\usepackage[czech]{babel}
\usepackage[utf8]{inputenc}
\usepackage{graphicx}
\usepackage{float}

\begin{document}
	\title{Semestrální práce z předmětu KIV/PPR}
	\subtitle{Prolomení šifry SkipJack}
	\author{Zdeněk Valeš}
	\date{22.11. 2018}
	\maketitle
	\newpage
	
	\section{Zadání}
	Vaším úkolem bude prolomit šifru SkipJack. Tuto šifru je výpočetně náročné prolomit hrubou silou, nicméně lze zkusit i sofistikovanější metody např. genetické a evoluční algoritmy. Abyste prolomení urychlili, lze referenční kód přepsat a vektorizovat na úrovni instrukcí, pomocí GPU, případně ho distribuovat pomocí MPI.
	
	Samostatná práce využije alespoň dvě z celkem tří možných technologií:
	
	\begin{itemize}
		\item Paralelní program pro systém se sdílenou pamětí - C++, popř. WinAPI
		\item Program využívající asymetrický multiprocesor - konkrétně x86 CPU a OpenCL kompatibilní GPGPU - C++ AMP
		\item Paralelní program pro systém s distribuovanou pamětí - C++ MPI
	\end{itemize}
	
	\section{Popis řešení}
	Cenová fce, jak to přibližně funguje
	
	\subsection{Diferenciální evoluce}
	Jak to cca funguje, který typ mutace jsem použil, hodnoty parametrů
	
	\subsection{Paralelizace}
	Vektory na mutaci ne-e. Použito TBB + OpenCL, popis paralelizace evoluce (parallel\_for)
	
	\section{Výsledky}
	
	\section{Závěr}
	
\end{document}
